\chapter*{简介}
\addcontentsline{toc}{chapter}{简介} \markboth{简介}{}

我们处理热力学物理的方法和入门物理课程中的传统不同。因此,我们提供了这个简介来展示在接下来的几个章节中我们将讲些什么。我们以这样的逻辑结构展示主线:在这个课程中,所有的物理从逻辑上。按照它们出现的顺序,在我们的叙述中,开头的章节是熵、温度、玻尔兹曼因子、化学势、吉布斯因子和分布函数。

熵是一个系统可处于的量子态的度量。一个封闭的系统可能会处于这其中的量子态的任意之一,并且({\itshape 我们假定})它们是等概率的。基本统计学元素,即基本逻辑假设是,系统要么处于要么不处于这些量子态,并且这个系统等可能地处于其中任意一种可能态相同的量子态。给定$g$个可能态\footnote{即上述“可能处于的量子态”,原文:accessible state ——译者注},熵定义为$\sigma=\ln g$。因此,这样定义的熵是能量$U$、粒子数$N$和体积$V$的函数,因为这些参数决定着$g$,当然其他参数也可能有贡献。对数的使用是出于数学上的便利:写出$10^{20}$比写出$\exp({10^{20}})$是要容易的,并且对于两个系统,$\sigma_1+\sigma_2$比$g_1g_2$更自然。

当两个有着各自特定能量的系统有热力学联系,它们可能会交换能量:它们的总能量保持为一个常数,但在它们各自能量上的约束会增加。能量的单向传输可能会增加乘积$g_1g_2$,后者代表着组合系统的可能态。基本假设认为,从能量分配上看,最终结果是使得可能态最大化的,即越多越好、可能性也越大。这一论述是熵增定律的核心,也是热力学第二定律的一般陈述。

